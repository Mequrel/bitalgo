\documentclass[
  fontsize=12pt  %
 ,english        % 
 ,headinclude    %
 ,headsepline    % line between head an document text
%,BCOR=12mm      % 
]{scrbook}       % twosided, A4 paper

\usepackage[T1]{fontenc}

\usepackage{polski}
\usepackage[utf8]{inputenc}
\usepackage[french,polutonikogreek,polish]{babel}

%\geometry{verbose,a4paper,tmargin=3cm,bmargin=2cm,lmargin=2cm,rmargin=2cm}


\usepackage{blindtext} % provides blindtext with sectioning

\usepackage{scrpage2}  % header and footer for KOMA-Script

\usepackage{graphicx}

\clearscrheadfoot                 % deletes header/footer
\pagestyle{scrheadings}           % use following definitions for header/footer
% definitions/configuration for the header
\rehead[]{\Large \textbf{BitAlgo Start}}        % equal page, right position (inner) 
\lohead[]{\Large \textbf{BitAlgo Start}}        % odd   page, left  position (inner) 
\cehead[]{Zadanie I:\\Rozwiązywanie równań}
\cohead[]{Zadanie I:\\Rozwiązywanie równań}
\lehead[]{\includegraphics[width=15mm]{logo.png}} % equal page, left (outer) position
\rohead[]{\includegraphics[width=15mm]{logo.png}}
% definitions/configuration for the footer
\cofoot[\pagemark]{\pagemark}     % odd   page, center position

\begin{document}
\vspace{50 mm}
\hspace{50 mm}
\newline

\par{\Large \textbf{Zadanie I: Rozwiązywanie równań}} \\ \\
\indent Twoim zadaniem jest wyliczenie pierwiastka równania:
 $$f(x) = 0$$
gdzie:
$$f(x) = Asin(Bx + C) + De^{(Ex + F)} + G|Hx + I| + J$$
Możesz założyć, że istnieje dokładnie jedno rozwiązanie na zadanym przedziale $[p; q]$.
\\ \\
\par{\Large \textbf{Format wejścia}} \\ \\
\indent W~pierwszej linii wejścia znajdują się dwie liczby wymierne $p$ i~$q$ $(p \leq q)$ - odpowiednio początek i~koniec przedziału. W~następnej linii znajduje się $10$ liczb wymiernych: $A, B, C, D, E, F, G, H, I, J$. Są to współczynniki funkcji. Możesz przyjąć, że wszystkie dane zmieszczą się w~typie double oraz że funkcja jest obliczalna i ciągła na szukanym przedziale, jak również, że przyjmuje przeciwne znaki na krańcach zadanego przedziału.
\\ \\
\par{\Large \textbf{Format wyjścia}} \\ \\
\indent Na standardowym wyjściu powinna znaleźć się jedna liczba $x$ - wartość pierwiastka równania.
Oczekiwana dokładność to $7$ miejsc po przecinku.
\\ \\
\par{\Large \textbf{Przykład}} \\ \\
\begin{tabular}{ p{7cm} p{7cm} }
	Dla danych wejściowych: \hspace{40mm}& Poprawną odpowiedzią jest \\
& \\

% define input here
-0.5 0.5 \newline
0 0 0 0 0 0 1 1 1 -1 \newline

&   
% define output here
0.0 \newline

\\

\end{tabular}

\end{document}


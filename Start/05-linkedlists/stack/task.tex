\documentclass[
  fontsize=12pt  %
 ,english        % 
 ,headinclude    %
 ,headsepline    % line between head an document text
%,BCOR=12mm      % 
]{scrbook}       % twosided, A4 paper

\usepackage[T1]{fontenc}

\usepackage{polski}
\usepackage[utf8]{inputenc}
\usepackage[french,polutonikogreek,polish]{babel}

%\geometry{verbose,a4paper,tmargin=3cm,bmargin=2cm,lmargin=2cm,rmargin=2cm}


\usepackage{blindtext} % provides blindtext with sectioning

\usepackage{scrpage2}  % header and footer for KOMA-Script

\usepackage{graphicx}

\clearscrheadfoot                 % deletes header/footer
\pagestyle{scrheadings}           % use following definitions for header/footer
% definitions/configuration for the header
\rehead[]{\Large \textbf{BitAlgo Start}}        % equal page, right position (inner) 
\lohead[]{\Large \textbf{BitAlgo Start}}        % odd   page, left  position (inner) 
\cehead[]{Zadanie R:\\Stos}
\cohead[]{Zadanie R:\\Stos}
\lehead[]{\includegraphics[width=15mm]{logo.png}} % equal page, left (outer) position
\rohead[]{\includegraphics[width=15mm]{logo.png}}
% definitions/configuration for the footer
\cofoot[\pagemark]{\pagemark}     % odd   page, center position

\begin{document}
\vspace{50 mm}
\hspace{50 mm}
\newline

\par{\Large \textbf{Zadanie R: Stos}} \\ \\
\indent Twoim zadaniem jest zaimplementowanie stosu korzystającego z~jednokierunkowej listy wiązanej. Twoje rozwiązanie powinno obsługiwać trzy operacje:
\begin{enumerate}
		\item $TOP$ - podaje jaka wartość jest na szczycie stosu.
		\item $POP$ - zdejmuje wartość ze szczytu stosu. Jeśli stos jest pusty operacja $POP$ nic nie robi.
		\item $PUSH$ $X$ - wstawia wartość $X$ na szczyt stosu.
\end{enumerate}
\\ \\
\par{\Large \textbf{Format wejścia}} \\ \\
\indent W~pierwszej linii wejścia znajduje liczba naturalna $z$ - liczba operacji na stoie $(z \leq 1000000)$. W~kolejnych $z$ liniach znajduje się lista operacji na stosie. Wszystkie przechowywane wartości to liczby całkowite mieszczące się w~type $int$.
\\ \\
\par{\Large \textbf{Format wyjścia}} \\ \\
\indent Dla każdej operacji $TOP$ program powinien wypisać pojedynczą liczbę - wartość ze szczytu stosu. Jeżeli stos jest pusty program powinien wypisać $EMPTY$.
\\ \\
\par{\Large \textbf{Przykład}} \\ \\
\begin{tabular}{p{7cm} p{7cm} }
	Dla danych wejściowych: \hspace{40mm}& Poprawną odpowiedzią jest \\
& \\

% define input here
10 \newline
PUSH 5 \newline
PUSH 7 \newline
TOP \newline
PUSH 2 \newline
TOP \newline
POP \newline
POP \newline
TOP \newline
POP \newline
TOP \newline

&   
% define output here
7 \newline
2 \newline
5 \newline
EMPTY \newline

\\

\end{tabular}

\end{document}

\documentclass[
  fontsize=12pt  %
 ,english        % 
 ,headinclude    %
 ,headsepline    % line between head an document text
%,BCOR=12mm      % 
]{scrbook}       % twosided, A4 paper

\usepackage[T1]{fontenc}

\usepackage{polski}
\usepackage[utf8]{inputenc}
\usepackage[french,polutonikogreek,polish]{babel}

%\geometry{verbose,a4paper,tmargin=3cm,bmargin=2cm,lmargin=2cm,rmargin=2cm}


\usepackage{blindtext} % provides blindtext with sectioning

\usepackage{scrpage2}  % header and footer for KOMA-Script

\usepackage{graphicx}

\clearscrheadfoot                 % deletes header/footer
\pagestyle{scrheadings}           % use following definitions for header/footer
% definitions/configuration for the header
\rehead[]{\Large \textbf{BitAlgo Start}}        % equal page, right position (inner) 
\lohead[]{\Large \textbf{BitAlgo Start}}        % odd   page, left  position (inner) 
\cehead[]{Zadanie K:\\Inwersje}
\cohead[]{Zadanie K:\\Inwersje}
\lehead[]{\includegraphics[width=15mm]{logo.png}} % equal page, left (outer) position
\rohead[]{\includegraphics[width=15mm]{logo.png}}
% definitions/configuration for the footer
\cofoot[\pagemark]{\pagemark}     % odd   page, center position

\begin{document}
\vspace{50 mm}
\hspace{50 mm}
\newline

\par{\Large \textbf{Zadanie K: Inwersje}} \\ \\
Twoim zadaniem jest zliczenie wszystkich inwersji w danym ciągu $a_i$. Inwersję definiujemy jako parę $(a_j, a_k)$ taką, że $j < k$ i $a_j > a_k$.
\\
Podpowiedź: wykorzystaj sortowanie przez scalanie.
\\ \\
\par{\Large \textbf{Format wejścia}} \\ \\
Pierwsza linia wejścia zawiera liczbe całkowitą $0 \leq n \leq 10^6$ - ilość elementów ciągu. W kolejnej linii znajduje się dokładnie $n$ liczb oddzielonych spacją, które reprezentują kolejne elementy ciągu $a_i$, $0 \leq a_i \leq 10^9$.
\\ \\
\par{\Large \textbf{Format wyjścia}} \\ \\
Na wyjściu powinna zostać wypisana jedna liczba, która reprezentuje łączną ilość inwersji w zadanym ciągu.
\\ \\
\par{\Large \textbf{Przykład}} \\ \\
\begin{tabular}{ p{7cm} p{7cm} }

  Dla danych wejściowych: \hspace{40mm}& Poprawną odpowiedzią jest \\
& \\

% define input here
5 \newline
4 3 1 5 3 \newline

&   
% define output here
5 \newline

\end{tabular}

Wyjaśnienie przykładu: inwersje w tym ciągu to: $(4,3), (4,1), (4,3), (3,1), (5,3)$.

\end{document}
\documentclass[
  fontsize=12pt  %
 ,english        % 
 ,headinclude    %
 ,headsepline    % line between head an document text
%,BCOR=12mm      % 
]{scrbook}       % twosided, A4 paper

\usepackage[T1]{fontenc}

\usepackage{polski}
\usepackage[utf8]{inputenc}
\usepackage[french,polutonikogreek,polish]{babel}

%\geometry{verbose,a4paper,tmargin=3cm,bmargin=2cm,lmargin=2cm,rmargin=2cm}


\usepackage{blindtext} % provides blindtext with sectioning

\usepackage{scrpage2}  % header and footer for KOMA-Script

\usepackage{graphicx}

\clearscrheadfoot                 % deletes header/footer
\pagestyle{scrheadings}           % use following definitions for header/footer
% definitions/configuration for the header
\rehead[]{\Large \textbf{BitAlgo Start}}        % equal page, right position (inner) 
\lohead[]{\Large \textbf{BitAlgo Start}}        % odd   page, left  position (inner) 
\cehead[]{Zadanie M:\\Dwie gry}
\cohead[]{Zadanie M:\\Dwie gry}
\lehead[]{\includegraphics[width=15mm]{logo.png}} % equal page, left (outer) position
\rohead[]{\includegraphics[width=15mm]{logo.png}}
% definitions/configuration for the footer
\cofoot[\pagemark]{\pagemark}     % odd   page, center position

\begin{document}
\vspace{50 mm}
\hspace{50 mm}
\newline

\par{\Large \textbf{Zadanie M: Dwie gry}} \\ \\
Św. Mikołaj podarował Maćkowi kupon umożliwiający kupno gier w popularnym serwisie internetowym o maksymalnej wartości $k$ złotych. Maciek chciałby kupić dwie gry. Nie chce by kupon się nawet częściowo zmarnował, więc ma zamiar kupić takie dwie gry, których łączna cena wynosi $k$ złotych. Znając ceny wszystkich gier w sklepie, podpowiedz Maćkowi czy w ogóle istnieje taka para gier w sklepie, która spełnia jego kryterium. 
\\ \\
Uwaga! Wykorzystaj w tym zadaniu algorytm sortowania std::sort z biblioteki standardowej C++ (lub odpowiednik dla innego języka). W przypadku wykorzystywania algorytmu binary search również wykorzystaj funkcję std::binary\_search lub std::lower\_bound.
\\
\par{\Large \textbf{Format wejścia}} \\ \\
Pierwsza linia wejścia zawiera dwie liczby całkowite $n$ oraz $k$, $0 \leq n \leq 10^6$, $0 \leq k \leq 2 \cdot 10^9$ - ilość gier w sklepie oraz ilość pieniędzy, które Maciek dostał od św. Mikołaja. W kolejnej linii znajduje się dokładnie $n$ liczb oddzielonych spacją - ceny kolejnych gier w sklepie. Ceny są liczbami całkowitymi należącymi do przedziału $[0; 10^9]$.
\\ \\
\par{\Large \textbf{Format wyjścia}} \\ \\
Na wyjściu powinien zostać wypisany napis `TAK` jeśli Maciek może kupić dwie gry w sklepie spełniające jego kryterium lub napis `NIE` w przeciwnym przypadku.
\\ \\
\par{\Large \textbf{Przykład}} \\ \\
\begin{tabular}{ p{7cm} p{7cm} }

  Dla danych wejściowych: \hspace{40mm}& Poprawną odpowiedzią jest \\
& \\

% define input here
10 13 \newline
9 8 11 20 3 1 9 5 9 13 \newline

&   
% define output here
TAK
\\

\end{tabular}
\\
\begin{tabular}{ p{7cm} p{7cm} }

  Dla danych wejściowych: \hspace{40mm}& Poprawną odpowiedzią jest \\
& \\

% define input here
10 2 \newline
9 8 11 20 3 1 9 5 9 13 \newline

&   
% define output here
NIE
\\

\end{tabular}
\end{document}
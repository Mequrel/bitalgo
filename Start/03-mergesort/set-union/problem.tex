\documentclass[
  fontsize=12pt  %
 ,english        % 
 ,headinclude    %
 ,headsepline    % line between head an document text
%,BCOR=12mm      % 
]{scrbook}       % twosided, A4 paper

\usepackage[T1]{fontenc}

\usepackage{polski}
\usepackage[utf8]{inputenc}
\usepackage[french,polutonikogreek,polish]{babel}

%\geometry{verbose,a4paper,tmargin=3cm,bmargin=2cm,lmargin=2cm,rmargin=2cm}


\usepackage{blindtext} % provides blindtext with sectioning

\usepackage{scrpage2}  % header and footer for KOMA-Script

\usepackage{graphicx}

\clearscrheadfoot                 % deletes header/footer
\pagestyle{scrheadings}           % use following definitions for header/footer
% definitions/configuration for the header
\rehead[]{\Large \textbf{BitAlgo Start}}        % equal page, right position (inner) 
\lohead[]{\Large \textbf{BitAlgo Start}}        % odd   page, left  position (inner) 
\cehead[]{Zadanie L:\\Suma zbiorów}
\cohead[]{Zadanie L:\\Suma zbiorów}
\lehead[]{\includegraphics[width=15mm]{logo.png}} % equal page, left (outer) position
\rohead[]{\includegraphics[width=15mm]{logo.png}}
% definitions/configuration for the footer
\cofoot[\pagemark]{\pagemark}     % odd   page, center position

\begin{document}
\vspace{50 mm}
\hspace{50 mm}
\newline

\par{\Large \textbf{Zadanie L: Suma zbiorów}} \\ \\
Twoim zadaniem jest wykonanie operacji unii na dwóch zadanych zbiorach.
\\ \\
Uwaga: nie korzystaj z gotowych bibliotek do sortowania i napisz własny algorytm sortowania przez scalanie (mergesort).
\\ \\
\par{\Large \textbf{Format wejścia}} \\ \\
Pierwsza linia wejścia zawiera dwie liczby całkowite $0 \leq n, m \leq 10^6$ - rozmiary dwóch zbiorów. W kolejnych dwóch liniach podane są elementy odpowiednio pierwszego i drugiego zbioru oddzielone pustymi znakami. Elementy zbiorów nie są większe od miliarda ($0 \leq a_i, b_i \leq 10^9$). 
\\ \\
\par{\Large \textbf{Format wyjścia}} \\ \\
Na wyjściu powinien zostać wypisany zbiór, który jest sumą zbiorów $a$ i $b$. Elementy zbioru muszą być wypisane w kolejności niemalejącej i jak przystało na zbiór, nie mogą zawierać duplikatów.
\\ \\
\par{\Large \textbf{Przykład}} \\ \\
\begin{tabular}{ p{7cm} p{7cm} }

  Dla danych wejściowych: \hspace{40mm}& Poprawną odpowiedzią jest \\
& \\

% define input here
5 6 \newline
9 3 5 7 1 \newline
2 3 4 9 8 7 \newline

&   
% define output here
1 2 3 4 5 7 8 9 \newline

\end{tabular}
\end{document}
\documentclass[
  fontsize=12pt  %
 ,english        % 
 ,headinclude    %
 ,headsepline    % line between head an document text
%,BCOR=12mm      % 
]{scrbook}       % twosided, A4 paper

\usepackage[T1]{fontenc}

\usepackage{polski}
\usepackage[utf8]{inputenc}
\usepackage[french,polutonikogreek,polish]{babel}

%\geometry{verbose,a4paper,tmargin=3cm,bmargin=2cm,lmargin=2cm,rmargin=2cm}


\usepackage{blindtext} % provides blindtext with sectioning

\usepackage{scrpage2}  % header and footer for KOMA-Script

\usepackage{graphicx}

\clearscrheadfoot                 % deletes header/footer
\pagestyle{scrheadings}           % use following definitions for header/footer
% definitions/configuration for the header
\rehead[]{\Large \textbf{BitAlgo Start}}        % equal page, right position (inner) 
\lohead[]{\Large \textbf{BitAlgo Start}}        % odd   page, left  position (inner) 
\cehead[]{Zadanie Q:\\Sudoku}
\cohead[]{Zadanie Q:\\Sudoku}
\lehead[]{\includegraphics[width=15mm]{logo.png}} % equal page, left (outer) position
\rohead[]{\includegraphics[width=15mm]{logo.png}}
% definitions/configuration for the footer
\cofoot[\pagemark]{\pagemark}     % odd   page, center position

\begin{document}
\vspace{50 mm}
\hspace{50 mm}
\newline

\par{\Large \textbf{Zadanie Q: Sudoku}} \\ \\
Twoim zadaniem jest napisanie programu, który rozwiązuje sudoku. Sudoku to znana łamigłówka logiczna, w której otrzymujemy planszę $9 \times 9$ podzieloną na 9 obszarów $3 x 3$. W polach planszy musimy wpisać liczby ze zbioru $[1, 2, \ldots, 9]$ tak aby żadna z tych liczb się nie powtórzyła w żadnej kolumnie i w żadnym wierszu, a także w żadnych z dziewięciu obszarów. Plansza jest już częściowo wypełniona liczbami. Twoim zadaniem jest wstawić pozostałe liczby nie naruszając zasad gry. Możesz założyć, że zawsze istnieje jakieś rozwiązanie dla każdej zadanej zagadki.
Twój program powinien być w stanie rozwiązać najtrudniejsze sudoku (17 pól początkowych) w ciągu maksymalnie 2 sekund. Zalecane jest poszukanie przykładowych sudoku i ich rozwiązań w sieci, poczynając od tych najprostszych. \\
\par{\Large \textbf{Format wejścia}} \\ \\
Początkowy stan tablicy zdefiniowany jest jako 81 liczb podzielonych na 9 wierszy. Wartość z zakresu $[1, 2, \ldots, 9]$ oznacza, że w polu mamy już ustawioną wartość, natomiast $0$ oznacza, że pole jest puste.
\\ 
\par{\Large \textbf{Format wyjścia}} \\ \\
Na wyjściu powinna zostać wypisana plansza w identycznym formacie jak na wejściu, z tym, że każde pole jest uzupełnione (nie ma zer).
\\ 
\par{\Large \textbf{Przykład}} \\ \\
\begin{tabular}{ p{7cm} p{7cm} }

  Dla danych wejściowych: \hspace{40mm}& Poprawną odpowiedzią jest \\
& \\

% define input here
0 0 0 0 0 0 0 0 0 \newline
0 0 7 8 3 0 9 0 0 \newline
0 0 5 0 0 2 6 4 0 \newline
0 0 2 6 0 0 0 7 0 \newline
0 4 0 0 0 0 0 8 0 \newline
0 6 0 0 0 3 2 0 0 \newline
0 2 8 4 0 0 5 0 0 \newline
0 0 0 0 9 6 1 0 0 \newline
0 0 0 0 0 0 0 0 0 \newline

&   
% define output here
2 9 4 1 6 5 8 3 7 \newline
6 1 7 8 3 4 9 5 2 \newline
3 8 5 9 7 2 6 4 1 \newline
5 3 2 6 8 1 4 7 9 \newline
7 4 1 2 5 9 3 8 6 \newline
8 6 9 7 4 3 2 1 5 \newline
9 2 8 4 1 7 5 6 3 \newline
4 7 3 5 9 6 1 2 8 \newline
1 5 6 3 2 8 7 9 4 \newline

\\

\end{tabular}
\end{document}
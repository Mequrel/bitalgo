\documentclass[
  fontsize=12pt  %
 ,english        % 
 ,headinclude    %
 ,headsepline    % line between head an document text
%,BCOR=12mm      % 
]{scrbook}       % twosided, A4 paper

\usepackage[T1]{fontenc}

\usepackage{polski}
\usepackage[utf8]{inputenc}
\usepackage[french,polutonikogreek,polish]{babel}

%\geometry{verbose,a4paper,tmargin=3cm,bmargin=2cm,lmargin=2cm,rmargin=2cm}


\usepackage{blindtext} % provides blindtext with sectioning

\usepackage{scrpage2}  % header and footer for KOMA-Script

\usepackage{graphicx}

\clearscrheadfoot                 % deletes header/footer
\pagestyle{scrheadings}           % use following definitions for header/footer
% definitions/configuration for the header
\rehead[]{\Large \textbf{BitAlgo Start}}        % equal page, right position (inner) 
\lohead[]{\Large \textbf{BitAlgo Start}}        % odd   page, left  position (inner) 
\cehead[]{Zadanie N:\\Kolejna permutacja}
\cohead[]{Zadanie N:\\Kolejna permutacja}
\lehead[]{\includegraphics[width=15mm]{logo.png}} % equal page, left (outer) position
\rohead[]{\includegraphics[width=15mm]{logo.png}}
% definitions/configuration for the footer
\cofoot[\pagemark]{\pagemark}     % odd   page, center position

\begin{document}
\vspace{50 mm}
\hspace{50 mm}
\newline

\par{\Large \textbf{Zadanie N: Kolejna permutacja}} \\ \\
Twoim zadaniem jest wyznaczenie kolejnej permutacji w porządku leksykograficznym dla zadanego ciągu. Przykładowo dla ciągu $3, 2, 1, 2$ kolejne permutacje w tym porządku to: $(1, 2, 2, 3), (1, 2, 3, 2), (1, 3, 2, 2), (3, 1, 2, 2), (3, 2, 1, 2), (3, 2, 2, 1)$.

W przypadku, gdy wejściowa permutacja jest ostatnią w porządku leksykograficznym, odpowiedzią jest permutacja pierwsza.
\\ \\
Uwaga! Nie wykorzystuj gotowej metody std::next\_permutation z biblioteki standardowej. Rozwiązania, które używają tej metody będą unieważniane.
\\
\par{\Large \textbf{Format wejścia}} \\ \\
Pierwsza linia wejścia zawiera jedną liczbę całkowitą $n \geq 1$ oznaczającą długość ciągu. W kolejnej linii znajduje się $n$ liczb, które definiują aktualną permutację ciągu. Elementy ciągu są liczbami całkowitymi należącymi do przedziału $[0; 10^9]$.
\\ \\
\par{\Large \textbf{Format wyjścia}} \\ \\
Na wyjściu powinno zostać wypisanych $n$ liczb - kolejne elementy ciągu w kolejności zdefiniowanej przez kolejną w porządku leksykograficznym permutację.
\\ \\
\par{\Large \textbf{Przykład}} \\ \\
\begin{tabular}{ p{7cm} p{7cm} }

  Dla danych wejściowych: \hspace{40mm}& Poprawną odpowiedzią jest \\
& \\

% define input here
10 \newline
3 4 3 2 1 2 4 3 2 1 \newline

&   
% define output here
3 4 3 2 1 3 1 2 2 4
\\

\end{tabular}
\\
\begin{tabular}{ p{7cm} p{7cm} }

  Dla danych wejściowych: \hspace{40mm}& Poprawną odpowiedzią jest \\
& \\

% define input here
4 \newline
4 3 2 1 \newline

&   
% define output here
1 2 3 4
\\

\end{tabular}
\end{document}
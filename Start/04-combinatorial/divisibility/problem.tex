\documentclass[
  fontsize=12pt  %
 ,english        % 
 ,headinclude    %
 ,headsepline    % line between head an document text
%,BCOR=12mm      % 
]{scrbook}       % twosided, A4 paper

\usepackage[T1]{fontenc}

\usepackage{polski}
\usepackage[utf8]{inputenc}
\usepackage[french,polutonikogreek,polish]{babel}

%\geometry{verbose,a4paper,tmargin=3cm,bmargin=2cm,lmargin=2cm,rmargin=2cm}


\usepackage{blindtext} % provides blindtext with sectioning

\usepackage{scrpage2}  % header and footer for KOMA-Script

\usepackage{graphicx}

\clearscrheadfoot                 % deletes header/footer
\pagestyle{scrheadings}           % use following definitions for header/footer
% definitions/configuration for the header
\rehead[]{\Large \textbf{BitAlgo Start}}        % equal page, right position (inner) 
\lohead[]{\Large \textbf{BitAlgo Start}}        % odd   page, left  position (inner) 
\cehead[]{Zadanie P:\\Dzielniki}
\cohead[]{Zadanie P:\\Dzielniki}
\lehead[]{\includegraphics[width=15mm]{logo.png}} % equal page, left (outer) position
\rohead[]{\includegraphics[width=15mm]{logo.png}}
% definitions/configuration for the footer
\cofoot[\pagemark]{\pagemark}     % odd   page, center position

\begin{document}
\vspace{50 mm}
\hspace{50 mm}
\newline

\par{\Large \textbf{Zadanie P: Dzielniki}} \\ \\
Twoim zadaniem jest obliczenie ile liczb w zadanym przedziale liczb całkowitych $[a; b]$ (przedział domknięty) jest podzielne przez chociaż jedną z zadanych $n$ liczb z ciągu $p$.
\\
\par{\Large \textbf{Format wejścia}} \\ \\
Pierwsza linia wejścia zawiera trzy liczby całkowite: $1 \leq n \leq 15$, $1 \leq a \leq b \leq 10^9$. W kolejnych $n$ liniach znajduje się po jednej liczbie $p_i$.
\\ \\
\par{\Large \textbf{Format wyjścia}} \\ \\
Na wyjściu powinna zostać wypisana jedna liczba - ilość liczb w zadanym przedziale, które są podzielne przez chociaż jedną z zadanych liczb.
\\ \\
\par{\Large \textbf{Przykład}} \\ \\
\begin{tabular}{ p{7cm} p{7cm} }

  Dla danych wejściowych: \hspace{40mm}& Poprawną odpowiedzią jest \\
& \\

% define input here
3 1 10000\newline
6 \newline
15 \newline
20 \newline

&   
% define output here
2333 \newline

\\

\end{tabular}
\end{document}